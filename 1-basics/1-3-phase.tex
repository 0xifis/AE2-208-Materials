\section{Phase} % (fold)
\label{sec:phase}
A phase is a region with uniform physical and chemical properties.

\subsection{Grain Boundaries} % (fold)
\label{sub:grain_boundaries}

A \textbf{single phase material} is usually polycrystalline (made up of millions of small crystals or grain). Each grain is region with the same crystal orientation. There exists a grain boundary between each grain.

The packing density at the grain boundary layer is very much reduced. This allows for much greater diffusion in the boundary plan than within the grain. As such, it is very easy for 'outsized' impurities to be found close to the grain boundaries.

The surface energy per unit area at the grain boundaries is
\begin{equation}
  Y_{gb} \approx 0.5 Jm^{-2}
\end{equation}
% subsection grain_boundaries (end)

\subsection{Interphase Boundaries} % (fold)
\label{sub:interphase_boundaries}
For \textbf{multi-phase materials}, there would be interphase boundaries between grains of different phases
\begin{itemize}
  \item the main phase - $\alpha$ particles \emph{(eg. random solution of B in A)}
  \item secondary phase - $\beta$ particles \emph{(eg. ordered solution of B \& A)}
\end{itemize}
and surface energy at the boundary is 
\begin{equation}
  Y_{\alpha\beta}
\end{equation}
This energy depends on the type of interphase boundary -
\begin{enumerate}
  \item \textbf{Coherent} - Lowest surface energy ($\approx0.05Jm^{-2}$); Because the structural and orientation change is nil, the surface energy is kept low.
  \item \textbf{Coherency strain} - The slightly different lattice spacing increases the surface energy.
  \item \textbf{Semi-coherent} - The strain increases as the particle grows but is relieved at points of dislocation.
  \item \textbf{Incoherent} - Large changes in the crystal structures in the grain results in high surface energy ($\approx0.5Jm^{-2}$).
\end{enumerate}

% subsection interphase_boundaries (end)
% section phase (end)

\section{Structure of Metals}
\subsection{Basic Definitions}

\textbf{Polymorphism}: material exist in more than one form or crystal structure

\textbf{Alloys}: metallic substance composed of two or more elements, they can be stronger than pure metals 

\textbf{Phase}: region of material with uniform physical and chemical properties, phase boundaries have an energy associated with them

\textbf{Isotropic}: here it means $\gamma$ doesn't depend on $\alpha$ and $\beta$ orientation

\subsection{Metal Crystal Structures} % (fold)
\label{sub:metal_crystal_structures}

\cimg{crystal_structures.png}

\subsubsection{Face Centred Cubic (FCC) - Aluminium, Copper, Lead, Nickel, Silver}
\begin{enumerate}
  \item Cubic structure with additional atoms at the centre of each face
  \item Highest packing density $74\%$
  \item Coordination number - 12
  \item Atoms/cell - 4
  \item Close packed planes - ${111} x 4$
  \item Close packed direction -$ <110> x 3$
  \item Total closed packed systems - 12
  \item Very ductile due to high no. of slip systems
\end{enumerate}

\subsubsection{Body Centred Cubic (BCC) - Iron}
\begin{enumerate}
  \item Cubic structure with additional atom at the centre of the cell
  \item Lower Packing density - $68\%$
  \item Coordination number - 8
  \item Atoms/cell - 2
  \item Close packed planes - 0
  \item Close packed direction -$ <111> x 3$
  \item Total closed packed systems - 0
\end{enumerate}

\subsubsection{Hexagonal Close Packed (HCP) - $\alpha$ Titanium, Magnesium, Zinc}
\begin{enumerate}
  \item Hexagonal structure with 3 planes separated by 3 atoms in the centre
  \item Highest packing density - $74\%$
  \item Coordination number - 12
  \item Atoms/cell - 6
  \item Close packed planes - ${0001} x 1$
  \item Close packed direction - $<1120> x 3$
  \item Total closed packed systems - 3
  \item Brittle due to low no. of closely packed systems
\end{enumerate}

\subsection{Polymorphism}
\cimg{polymorphism.png}

\begin{table}[h!]
\centering
\caption{Polymorphism in aerospace materials}
\label{polymorphism-metals}
\begin{tabular}{|l|l|}
\hline
\textbf{Magnesium} & hcp      \\ \hline
\textbf{Titanium}  & hcp, bcc \\ \hline
\textbf{Iron}      & bcc, fcc \\ \hline
\textbf{Cobalt}    & hcp, fcc \\ \hline
\textbf{Nickel}    & fcc      \\ \hline
\textbf{Aluminium} & fcc      \\ \hline
\end{tabular}
\end{table}

However, some metals may exhibit different crystal structures at different temperature ranges, aka polymorphism. 

In some case, when molten metals are cooled really fast, there is no time for the randomly arranged atoms in the liquid state to switch into the orderly arrangement of a solid crystal. This forms a 'glassy' or 'amorphous' solid. The solid would have a 'frozen-in liquid structure'.

% subsection metal_crystal_structures (end)

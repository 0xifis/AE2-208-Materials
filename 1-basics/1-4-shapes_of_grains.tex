\section{Shapes of Grains}
\subsection{In Single Phase Metal} % (fold)
\label{sub:in_single_phase_metal}
Grain shapes and sizes are determined by the physical processes in which the metal is processed.

However, in the absence of external effects, the shape of grains is the shape that minimises the total interfacial energy, $E$
\begin{equation}
  E = \gamma_{gb}\cdot A
\end{equation}
If $\gamma_{gb}$ is isotropic, the lowest interfacial energy would be the that with the lowest interfacial area.
\begin{enumerate}
  \item A polycrystalline metal is 3 dimensional $\rightarrow$ planes meet at $120\degree$
  \item The grains need to pack together to fill space
  \item Most basic grain shape is tetrakaidecahedron (14 faces)
\end{enumerate}

% subsection in_single_phase_metal (end)
\subsection{In Two Phase materials} % (fold)
\label{sub:in_two_phase_materials}
\cimg{grain_shapes.png}
Considering $\beta$-phase grain in $\alpha$,
\begin{enumerate}
  \item If $\gamma_{\alpha\beta} = \gamma_{bg}$, $\beta$ will form tetrakaidecahedron grains.
  \item If $\gamma_{\alpha\beta}$ is isotropic, the lowest interfacial energy, $E$, occurs when area, $A$, is minimised. Thus, $\beta$ will form \textbf{spherical grains}.
  \item If $\gamma_{\alpha\beta}$ depends on orientation, $\beta$ will form \textbf{plate-like grains} oriented in the direction of lowest interfacial energy.
  \item If $\beta$ forms at $\alpha$ boundaries, $\beta$ will form lens-shaped grains where the angle, $\theta$ is
  \begin{equation}
    2\gamma_{\alpha\beta}\cos\theta = \gamma_{gb}
  \end{equation}
  If $\gamma_{\alpha\beta} < \frac{1}{2}\gamma_{gb}$, $\cos\theta = 1 \rightarrow \theta = 0$. Thus, forming a thin layer of $\beta$ around the grain boundary.
\end{enumerate}
% subsection in_two_phase_materials (end)
\section{Phase Diagrams} % (fold)
\label{sec:phase_diagrams}

\subsection{Equilibrium Constitution} % (fold)
\label{sub:equilibrium_constitution}
The equilibrium constitution of a sample is its constitution when there is no further tendency for its constitution to change with time when at a given constant temperature and pressure. 

The independent constitution variables are $T, p$ and composition.
% subsection equilibrium_constitution (end)

\subsection{Reading Phase Diagrams} % (fold)
\label{sub:reading_phase_diagrams}

Every alloy pair has a phase digram for it. The phase digram below shows a case of complete solid solubility. Cu can dissolve unlimited in Ni and vice versa because they are similarly sized and have FCC structure.

\cimg{basic_phase_diagram}

\subsubsection{Finding the proportion of each phase}
For the multi-phase segment, the proportion by weight of $\alpha$, $f_\alpha$, in the mixture is given by
\begin{equation}
  f_\alpha = \frac{C_0-C_L}{C_\alpha-C_L}
\end{equation}
where $C$ represents the composition of Ni in the various phases.

\subsubsection{Important points on the phase diagram}
\cimg{phase_diagram}
\textbf{Eutectic Point}: Point where the alloy system changes to pure solid state from liquid state with no transition in between

\textbf{Eutectoid Point}: Point where a single phase solid transforms to a 2 phase solid.
% subsection reading_phase_diagrams (end)

% section phase_diagrams (end)

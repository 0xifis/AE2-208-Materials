\section{Aluminium Alloys} % (fold)
\label{sec:aluminium_alloys}

Aluminium alloys are very interesting in aerospace because of their \textbf{lightweight, high strength and good corrosion resistance}.

\subsection{Strengthening Mechanisms} % (fold)
\label{sub:strengthening_mechanisms}

\begin{enumerate}
  \item \textbf{Solution Hardening} - Internal stresses from dissolved solid solute atoms
  \item \textbf{Grain Refinement} - Reducing the grain size (Hall–Petch relationship)
  \item \textbf{Precipitation Hardening} - Internal stresses from precipitates
  \item \textbf{Work Hardening} - Immobile dislocations used to block mobile dislocations
\end{enumerate}

% subsection strengthening_mechanisms (end)

\subsection{Aluminium Series} % (fold)
\label{sub:aluminium_series}

\begin{table}[H]
\centering
\label{aluminium_series}
{\renewcommand{\arraystretch}{2}
\begin{tabular}{|c|c|c|}
\hline
\textbf{Series} & \textbf{Composition} & \textbf{Phases} \\ \hline
\textbf{2xxx}   & Al-Cu(-Mg) series    &  \pbox{20cm}{$\theta’ (Al_2Cu)$ \\ $S’ (Al_2CuMg)$} \\ \hline
\textbf{6xxx}   & Al-Mg-Si    & \pbox{20cm}{$\lambda’ (Al_5Cu_2Mg_8Si_7)$ \\ $\beta’ (Mg_2Si)$} \\ \hline
\textbf{7xxx}   & Al-Zn-Mg(-Cu)  & \pbox{20cm}{$\eta/\eta’ (MgZn_2)$\\ $T’ (Mg_{32}(Al,Zn)_{49})$} \\ \hline
\textbf{8xxx}   & Al(-Li)   & \pbox{20cm}{$\delta’ (Al_3Li)$ \\ $T_1 (Al_2CuLi)$} \\ \hline
\end{tabular}}
\end{table}

% subsection aluminium_series (end)

\subsection{Duralumin ($Al_4Cu$)} % (fold)
\label{sub:duralumins}

Duralumin is an aluminium alloy containing \wtcu{4}.

\cimg{duralumins}

Just like with steel (size of cementite precipitate), the rate of cooling affects the size of the $Cu_2Al$ precipitate.

\subsubsection{Cooled Too Slowly} % (fold)
\label{ssub:cooled_too_slowly}

\begin{itemize}
  \item driving force for the precipitation of $CuAl_2$ rate of nucleation are too small
  \item large, spaced-apart precipitates are formed
  \item moving dislocations can easily avoid the precipitates
  \item final alloy will be soft
\end{itemize}
% subsubsection cooled_too_slowly (end)

\subsubsection{Cool Too Fast} % (fold)
\label{ssub:cool_too_fast}

\begin{itemize}
  \item miss the nose of the TTT curve
  \item $Cu_Al2$ precipitate is not formed
\end{itemize}
% subsubsection cool_too_fast (end)

\subsubsection{Optimal Cooling Rate} % (fold)
\label{ssub:optimal_cooling_rate}


Hence, a controlled cooling rate is required to get
\begin{enumerate}
  \item fine-scaled precipitates of $Cu_Al2$
  \item distributed evenly throughout the $\alpha$-Al grains
\end{enumerate}

\cimg{alum_cooling}

\cimg{alum_ttt}

\begin{enumerate}
  \item After solidification, the alloy is held just below the eutectic temperature,$~550 \degree C$, to allow the solid solution to homogenise.
  \item From $550 \degree C$, Aluminium is quenched rapidly down (miss nose of TTT) to room temperature. 
  \item Reheated and held at $150 \degree C$ for 100 hours to allow $Cu$ atoms from the supersaturated $\alpha$ grains to form fine $CuAl_2$ precipitates.
\end{enumerate}
% subsubsection optimal_cooling_rate (end)

% subsection duralumins (end)

\newpage
\subsection{Ageing Process} % (fold)
\label{sub:ageing}

\subsubsection{0: Super Saturated Solid Solution} % (fold)
\label{ssub:solid_solution}

\cimg[200pt]{ageing_1}


\begin{itemize}
  \item At the start of ageing the alloy is a supersaturated solid solution (substitutional) of homogeneously distributed $Cu$ atoms.
\end{itemize}

% subsubsection solid_solution_hardening (end)

\subsubsection{1: Formation of GP Zones} % (fold)
\label{ssub:gp_zones}

\cimg[200pt]{ageing_2}

\begin{itemize}
  \item $Cu$ atoms dissolve as interstitial solute particles between $Al$ atoms and nucleate homogeneously to form disc-shaped 'Guinier-Preston' zones.
\end{itemize}

% subsubsection gp_zones (end)

\subsubsection{3: Growth of GP Zones} % (fold)
\label{ssub:growth_of_gp_zones}

\cimg[200pt]{ageing_3}

\begin{itemize}
  \item Some GP zones grow to form nucleation sites for metastable phase, $\theta''$.
  \item The remaining GP zones dissolve and transfer $Cu$ to the growing $\theta''$ phase through diffusion.
  \item Although, $\theta''$ disc faces are coherent with the Al-matrix, the disc edges create large coherency strains.
\end{itemize}

% subsubsection growth_of_gp_zones (end)

\subsubsection{3: Formation of $\theta'$ Phase} % (fold)
\label{ssub:formation_of_thata}

\cimg[200pt]{ageing_4}

\begin{itemize}
  \item Another metastable phase, $\theta'$, nucleate at matrix dislocations at the expense of $\theta''$
  \item The  remaining $\theta''$ precipitates all dissolve and transfer $Cu$ to the growing  $\theta'$.
  \item Now the disc faces are still perfectly coherent with the matrix but the disc edges are now \textbf{incoherent}. Thus, neither faces nor edges show coherency strain.
\end{itemize}

% subsubsection formation_of_theta (end)

\subsubsection{4: Formation of equilibrium phase, $\theta$} % (fold)
\label{ssub:formation_of_equilibrium_phase_}

\cimg[200pt]{ageing_5}

\begin{itemize}
  \item Equilibrium $CuAl_2$ ($\theta$) nucleates at grain boundaries and at  $\theta'$–matrix interfaces.
  \item The  remaining $\theta'$ precipitates all dissolve and transfer $Cu$ to the growing $\theta$ phase through diffusion.
  \item The $\theta$ phase has very different lattice parameters and is completely incoherent with the $\alpha$ matrix.
  \item Hence, it forms a rounded shape.
\end{itemize}

% subsubsection formation_of_equilibrium_phase_ (end)
% subsection ageing (end)


% section aluminium_alloys (end)
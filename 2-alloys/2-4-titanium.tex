\section{Titanium Alloys} % (fold)
\label{sec:titanium_alloys}

Titanium alloys form another group of useful alloys that is commonly used in aerospace.

\subsubsection{Advantages of Using Titanium} % (fold)
\label{ssub:advantages_of_using_titanium}

\begin{itemize}
  \item Low density - $4.5 g cm^{-3}$
  \item Excellent corrosion resistance
  \item High specific strength
  \item Good compatibility with CFRP structures
  \item Excellent properties at elevated temperatures
\end{itemize}
% subsubsection advantages_of_using_titanium (end)

\subsubsection{Disadvantages of Using Titanium} % (fold)
\label{ssub:disadvantages_of_using_titanium}

\begin{itemize}
  \item Expensive to machine
  \item Low wear resistance
  \item Difficult to form
  \item Pick up oxygen and nitrogen above $500\degree C$
\end{itemize}
% subsubsection disadvantages_of_using_titanium (end)

\subsection{Phases and $\alpha/\beta$-Ti stabilisers} % (fold)
\label{sub:phases_and_}

\cimg{ti_phases}
\cimg{ti_stabilisers}
Different solutes have different effects on the phase transformation temperature for Ti.
 % subsection phases_and_ (end)

There are 5 groups of Ti-alloys

\subsection{$\alpha$-Ti Alloys} % (fold)
\label{sub:alpha_alloys}
\begin{itemize}
  \item Equilibrium constitution at rtp is $100\% \alpha$-Ti
  \item eg. Ti-5Al-2.5Sn
\end{itemize}

\subsubsection{Strengthening Mechanisms} % (fold)
\label{ssub:strengthening_mechanisms}

\begin{enumerate}
  \item Solid solution strengthening due to oxygen
  \item Grain boundary strengthening due to the small grain size (Note: Ductility increases too)
  \item Work hardening through cold work to help give high dislocation density
\end{enumerate}

% subsubsection strengthening_mechanisms (end)
% subsection alpha_alloys (end)

\subsection{Near $\alpha$-Ti Alloys} % (fold)
\label{sub:near_}

\begin{itemize}
  \item eg. Ti-5Al-2.5Sn
\end{itemize}

\subsubsection{Diffusive Formation} % (fold)
\label{ssub:formation}

\begin{enumerate}
  \item Shape by casting, rolling etc.
  \item Heat treat in the $\alpha-\beta$ region and then slow cooled
  \item The hot $\alpha$ plates form between $\beta$ plates on cooling (diffusive)
  \item Gives great creep resistance but poor fatigue resistance
\end{enumerate}

% subsubsection formation (end)

\subsubsection{Displacive Formation} % (fold)
\label{ssub:displacive_formation}

\begin{enumerate}
  \item Shape by casting, rolling etc.
  \item Heat treat in the $\beta$ region and then quench
  \item 'Basket weave' structure of $\alpha$ plates will form between $\beta$ plates
  \item Gives decent creep resistance but good fatigue resistance
  \item Good at crack deflection - high convoluted crack path
\end{enumerate}

% subsubsection displacive_formation (end)

\subsubsection{Strengthening Mechanisms} % (fold)
\label{ssub:strengthening_mechanisms}

\begin{enumerate}
  \item Solid Solution strengthening due to large amount of solute
  \item Grain boundary strengthening due to small grain size and $\alpha-\beta$ interfaces
\end{enumerate}

% subsubsection strengthening_mechanisms (end)
% subsection near_ (end)

\subsection{$\alpha$-$\beta$-Ti Alloys} % (fold)
\label{sub:}
\begin{itemize}
  \item eg. Ti-6Al-4V , Ti-6Al-2Sn-4Zr-6Mo
\end{itemize}

\subsubsection{$\beta$-region Formation} % (fold)
\label{ssub:}

\begin{enumerate}
  \item Shape by casting, rolling etc.
  \item \textbf{Either} slow cool to get basket weave structure of $\alpha$ plates between $\beta$
  \item \textbf{or} quench to transform $\beta$ into martensitic form of $\alpha$-Ti
\end{enumerate}

% subsubsection  (end)

\subsubsection{$\alpha$-$\beta$-region Formation (slow cooled)} % (fold)
\label{ssub:subsubsection_name}

\begin{enumerate}
  \item Shape by casting, rolling etc.
  \item Slow cool
  \item Large white crystal grains form which were originally $\alpha$  from the $\alpha$-$
  \beta$ region
  \item Fine white $\alpha$ between plates of $\beta$ formed on cooling
\end{enumerate}

% subsubsection subsubsection_name (end)

\subsubsection{$\alpha$-$\beta$-region Formation (quenched)} % (fold)
\label{ssub:subsubsection_name}

\begin{enumerate}
  \item Shape by casting, rolling etc.
  \item Quench
  \item Original $\alpha$ remains, $\beta$ transforms into martensitic $\alpha$-Ti
  \item Age harden to transform martensite to fine structure of $\alpha$ and $\beta$
\end{enumerate}
% subsection  (end)


\subsection{near-$\beta$-Ti Alloys} % (fold)
\label{sub:}
\begin{itemize}
  \item eg. Ti-10V-2FE-3Al , Ti-5V-5Mo-5Al-6Mo-3Cr\end{itemize}

\subsubsection{Formation} % (fold)
\label{ssub:}

\begin{enumerate}
  \item Shape by casting, rolling etc.
  \item Quench
  \item $\beta$ transforms into martensitic form $\alpha$-Ti
  \item Age harden to produce fine structure of $\alpha$ and $\beta$
\end{enumerate}

\subsubsection{Strengthening Mechanisms} % (fold)
\label{ssub:strengthening_mechanisms}
\begin{itemize}
  \item Solid solution strengthening (highly alloyed)
  \item Grain boundary strengthening (small grain sizes)
  \item Precipitation strengthening
\end{itemize}

% subsubsection strengthening_mechanisms (end)
% subsubsection  (end)

\subsection{$\beta$-Ti Alloys} % (fold)
\label{sub:}
\begin{itemize}
  \item eg. Ti-13V-11Cr-3Al , Ti-15V-3Cr-3Al-3Sn\end{itemize}

\subsubsection{Formation} % (fold)
\label{ssub:}

\begin{enumerate}
  \item Shape by casting, rolling (elevated temperature) etc.
  \item Solution heat treatment at $750\degree C$-$850 \degree C$ (dissolve $\alpha$-Ti)
  \item Quench
  \item Age harden to produce fine structure of $\alpha$ precipitates in $\beta$
\end{enumerate}

\subsubsection{Strengthening Mechanisms} % (fold)
\label{ssub:strengthening_mechanisms}
\begin{itemize}
  \item Solid solution strengthening (highly alloyed)
  \item Grain boundary strengthening (small grain sizes)
  \item Precipitation strengthening
\end{itemize}

% subsubsection strengthening_mechanisms (end)
% subsubsection  (end)

\subsection{Summary of Ti-alloys} % (fold)
\label{sub:summary_of_ti_alloys}

\cimg{ti_summary}


% subsection summary_of_ti_alloys (end)

% section titanium_alloys (end)
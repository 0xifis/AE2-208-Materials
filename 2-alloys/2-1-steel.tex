\section{Steel} % (fold)
\label{sec:steel}

\cimg{steel_phase}

\subsection{Types of Reactions} % (fold)
\label{sub:types_of_reactions}

\subsubsection{Eutectic Reaction}
\begin{equation}
  L(Fe) \rightarrow \gamma(Fe) + Fe_3C
\end{equation}
\subsubsection{Eutectoid Reaction}
\begin{equation}
  \gamma(Fe) \rightarrow \alpha(Fe) + Fe_3C
\end{equation}
% subsection types_of_reactions (end)

\subsection{Phases in Steel} % (fold)
\label{sub:phases_in_steel}

\cimg[250pt]{phases_steel}

% subsection phases_in_steel (end)

\subsection{Diffusive Transformation} % (fold)
\label{sub:diffusive_transformation}

Diffusive transformation occurs as the material is slow cooled \emph{(or normalised)}. DCT is explained in Section \ref{sub:diffusion_controlled_phase_transformation}.

\subsubsection{In Pure Iron} % (fold)
\label{ssub:subsubsection_name}

\cimg[250pt]{iron_dct_pure}

In the case of pure iron (ie. \wtc{0}), the process is very straightforward. At $914\degree C$, the $\alpha$ grains nucleate at the grain boundaries of the $\gamma$ grains.

% subsubsection subsubsection_name (end)

\subsubsection{In Eutectoid Steel} % (fold)
\label{ssub:in_eutectoid_steel}

The eutectoid point in the steel system occurs at $723\degree C$ and at the eutectoid composition (\wtc{0.8}).
\begin{itemize}
  \item The austenite (containing \wtc{0.8})  changes into ferrite (containing almost no C) and cementite (containing \wtc{6.7}).
  \item The carbon atoms must diffuse from regions of high concentrations to low concentrations as the fcc austenite transforms into bcc ferrite and cementite lattices.
  \item Nuclei of small plates of ferrite and cementite form at the grain boundaries of the austenite and carbon diffusion occurs at a local scale.
  \item Then, these plates grow in tandem, consuming more austenite as they go.
  \item As a result, grains containing alternate plates of ferrite and cementite form.
  \item This ferrite-cementite micro-structure is called pearlite.*
\end{itemize}

Only some commercial steels have a eutectoid composition such as steel for railway track (pearlite steel).

\emph{* Pearlite is not a phase but a two-phase micro-structure}

\cimg[250pt]{iron_dct_eutectoid}

% subsubsection in_eutectoid_steel (end)


\subsubsection{In Hypo-Eutectoid Steel} % (fold)
\label{ssub:in_hypo_eutectoid_steel}

Most carbon steels are 'hypo-eutectoid', containing less than \wtc{0.8}. 
\begin{itemize}
  \item Mild Steel - \wtc{0.1-0.2}
  \item Medium Steel - \wtc{0.4}
\end{itemize}

\cimg[250pt]{iron_dct_hypo}

Consider slow-cooling a \wtc{0.3} steel, starting from $900 \degree C$ austenite with a solid solution of C.

\begin{itemize}
  \item At $820 \degree C$, we enter the austenite-ferrite region where the ferrite starts by nucleating on the austenite grain boundaries.
  \item The ferrite continues to discard C into the austenite.
  \item At just above $723 \degree C$, the proportion of austenite to ferrite is roughly $1:1$
  \item At this point, the austenite which is at \wtc{0.8} is roughly above the eutectoid point.
  \item Further cooling to just below $723 \degree C$causes the austenite to decompose rapidly to form the pearlite micro-structure. TADA!
\end{itemize}

% subsubsection in_hypo_eutectoid_steel (end)

\subsubsection{In Hyper-Eutectoid Steel} % (fold)
\label{ssub:in_hyper_eutectoid_steel}

The main difference between hyper-eutectoid steel and hypo-eutectoid steel is that, in this case, the $Fe_3C$ is nucleated first. Then the austenite's composition reduces until the eutectoid composition (at $723 \degree C$) before rapidly decomposing into pearlite.

\cimg[250pt]{iron_dct_hyper}

% subsubsection in_hyper_eutectoid_steel (end)

\emph{Good reference in Materials, Ashby M. - $GL2-48$ to $GL2-53$}

\subsubsection{Mechanical Properties of Normalised Steel} % (fold)
\label{ssub:mechanical_properties_of_normalised_steel}

\begin{itemize}
  \item Yield strength an tensile strength increases linearly with carbon content
  \item $Fe_3C$ acts as a strengthening base
  \item Proportion of $Fe_3C$ in steel is linear in carbon concentration
  \item However, ductility decreases because the $\alpha-Fe_3C$ interfaces in pearlite are good at nucleating cracks
\end{itemize}

% subsubsection mechanical_properties_of_normalised_steel (end)

% subsection diffusive_transformation (end)

\subsection{Displacive Transformation} % (fold)
\label{sub:displacive_transformation}

If the austenite is cooled rapidly (quenched), it is possible to transform the steel without diffusion but with displacive transformation. This forms, bct martensite, instead of bcc ferrite which has a lower carbon content.

\cimg[250pt]{iron_dt}

As seen from the above figure, if the carbon content of the austenite is increased, the cooling rate required could reduced doe to the withdrawal of the 'nose' in the TTT plot.

\subsubsection{Mechanical Properties of Martensite} % (fold)
\label{ssub:mechanical_properties_of_martensite}

\begin{itemize}
  \item Hardness of martensite increases with carbon content
  \item The carbon in martensite distorts the bcc martensite lattice, making it tetragonal (body-centred tetragonal). This distortion increases linearly with the amount of dissolved carbon
  \item Martensite is very brittle because the dislocations cannot move through the strained bct crystal easily
\end{itemize}

% subsubsection mechanical_properties_of_martensite (end)

% subsection displacive_transformation (end)

\subsection{Tempering Martensite} % (fold)
\label{sub:tempering_martensite}

With diffusion controlled transformation, we get a nice mixed ferrite-pearlite micro-structure, which has excellent strength and toughness. However, quenching at a high rate traps carbon in a supersaturated solution. Hence, tempering ($500-600\degree C$) of the steel is will provide the carbon atoms enough energy to diffuse out and form fine-scale precipitates of cementite. 

\cimg[250pt]{tempering}

As the carbon atoms leave to form the cementite, the martensite lattice will relax to form the undistorted bcc structure and ductility will go up to an acceptable range as a result. However, hardness and yield stress will go down. The net effect is that yield stress is about an order 2 or 3 higher than but ductility is comparable to that of normalised carbon steel.

% subsection tempering_martensite (end)

\subsection{Alloying Steel} % (fold)
\label{sub:alloying_steel}


We have seen earlier that quenched Martensite is usually hardened and provides very good properties for use. However, the key requirement here is to cool it fast enough.

Rapid cooling or quenching is usually done by placing red hot iron into a bath of cold water or oil. However, what if the component is large? So large that it is difficult for heat from deep within to be lost?

Furthermore, as cooling occurs, shrinkage stresses will put the untempered Martensite as risk of cracking.
The solution to this problem is then to use alloying. By introducing alloys, the entire TTT curve can be shifted to the right so that the quenching / rapid cooling process can take place at a slower rate.

Many low alloy steels have been developed with superior hardenability – the ability to form Martensite in thick sections when quenched.

\cimg[250pt]{alloying_steel}

As seen, adding Mo, Mn, Cr and/or Ni helps to shift the TTT curve to the right, enabling Martensite to form even at slow cooling rates. 

% subsection alloying_steel (end)

% section steel (end)

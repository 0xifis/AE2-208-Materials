\section{Magnesium Alloys} % (fold)
\label{sec:magnesium_alloys}

Magnesium alloys form another group of useful alloys that is commonly used in aerospace.

\subsubsection{Advantages of Using Magnesium} % (fold)
\label{ssub:advantages}

\begin{itemize}
  \item Lightest structural metal - $1783 kg m^{-3}$
  \item Precipitation harden-able like Aluminium
  \item Good creep resistance up to $250\degree C$
\end{itemize}
% subsubsection advantages (end)

\subsubsection{Disadvantages of Using Magnesium} % (fold)
\label{ssub:disadvantages_of_using_magnesium}

\begin{itemize}
  \item Low melting point compared to Ti and Ni - $600\degree C$
  \item Low stiffness - $45GPa$
  \item Single dislocation slip plane (HCP) at RTP
  \item Bad corrosion resistance
\end{itemize}
% subsubsection disadvantages_of_using_magnesium (end)

\subsection{Cast Alloys - AZ31(Mg-3Al-0.7Zn)} % (fold)
\label{sub:az31}

\cimg{az31}
Al and Mg alloys are very similar although they have different lattice structures - Mg (hcp), Al (fcc).

\subsubsection{Formation} % (fold)
\label{ssub:formation}

\begin{enumerate}
  \item Heat treatment of the solid solution at $400\degree C$.
  \item Quench the supersaturated solid solution to get $Al_{12}Mg_{17}$
  \item $Al_{12}Mg_{17}$ will nucleate on  $(0001)_{Mg}$
  \item Work hardening by cold rolling to strengthen
\end{enumerate}
% subsubsection formation (end)
% subsection az31 (end)

\subsection{Wrought Alloys - WE43()} % (fold)
\label{sub:we_alloys_}
This alloy has the best properties of all $Mg$ alloys but they are so damn freaking expensive.

\begin{table}[H]
\centering
{\renewcommand{\arraystretch}{2}
\begin{tabular}{|c|c|c|}
\hline
\textbf{Alloy} & \textbf{Composition} & \textbf{Phases} \\ \hline
\textbf{WE43}   & Mg-4Y-3Nd-Zr   & \pbox{20cm}{$Mg_{41}Nd_5$\\ $Mg_{24}Y_5$ \\ $MgY$ \\ $\beta (Mg_{12}YNd)$} \\ \hline
\end{tabular}}
\end{table}

\subsubsection{Precipitation Hardening} % (fold)
\label{ssub:precipitation_hardening}

\begin{equation}
  SSSS \rightarrow \beta'' \rightarrow \beta' \rightarrow \beta_1 \rightarrow \beta
\end{equation}

\begin{itemize}
  \item $\beta$'' - hexagonal, $Mg_3Y_{0.85}Nd_{0.15}$, coherent with $Mg$
  \item $\beta$' - orthorhombic, $Mg_{24}Y_{2}Nd_{3}$, semi-coherent with $Mg$
  \item $\beta_1$ - FCC
  \item $\beta$' - FCC, $Mg_{12}YNd$, incoherent with $Mg$
\end{itemize}

% subsubsection precipitation_hardening (end)

\subsubsection{Role of Zr} % (fold)
\label{ssub:role_of_zr}
\begin{itemize}
  \item Zr promotes nucleation of Mg grains during solidification
  \item Zr particles are very similar crystals to Mg and act as a 'template' for Mg
  \item Smaller grain boundaries promote better grain boundary strengthening and retards crack propagation.
\end{itemize}
% subsubsection role_of_zr (end)

\subsubsection{Role of Y and Nd} % (fold)
\label{ssub:role_of_y_and_nd}

\begin{itemize}
  \item In Mg-Y-Nd alloys, 4\% Y and 3\% Nd can be completely dissolved at high T and then quenched to give a SSSS
  \item Ageing is done next after quenching
  \item Peak strength occurs when there is a fine disperion of $\beta$' and $\beta_1$ or $\beta$
  \item The small amounts of $\beta_1$ or $\beta$' disrupts dislocation motion at elevated T.
  \item $\beta$ particles formed at the grain boundaries prevents grain boundary sliding at elevated T.
\end{itemize}
% subsubsection role_of_y_and_nd (end)
% subsection we_alloys_ (end)
% section magnesium_alloys (end)

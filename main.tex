\documentclass{summary_notes}
\include{custom-commands}

%--------------------------------------------------

\begin{document}

%--------------------------------------------------
% Titlepage
\title{\bf Revision Notes for AE2-208 Materials}
\author{Vishnu R Menon\\ 
\small{Department of Aeronautics, Imperial College London}}
\makeheader{AE2-208 Materials | Revision Notes}{Vishnu R Menon}
\maketitle
\tableofcontents
\newpage
%--------------------------------------------------


%--------------------------------------------------
% Chapter 1

\chapter{Basics}
\section{Structure of Metals}
\subsection{Basic Definitions}

\textbf{Polymorphism}: material exist in more than one form or crystal structure
\textbf{Alloys}: metallic substance composed of two or more elements, they can be stronger than pure metals 
\textbf{Phase}: region of material with uniform physical and chemical properties, phase boundaries have an energy associated with them\\
\textbf{isotropic}: here it means $\gamma$ doesn't depend on $\alpha$ and $\beta$ orientation\\


\textbf{Total interfacial energy}: 
\begin{align}
    E= \gamma_{ab}*Area 
    \end{align}

\vspace{0.25cm}
-------------
\vspace{0.25cm}

A structure is defined by : bonds (metallic, covalent..) and crystal packing (fcc,bcc..)

Now, properties of the material can be:
\begin{itemize}
\item structure sensitive (yield's strength, ductility..)
\item structure insensitive (Young's Modulus, density..)
\end{itemize}

\underline{About Alloys and how to get them}:\\
\\
You have either\\

\textbf{Substitutional} (replace some atoms of one material by the other material, the arrangement can be random, ordered or clustered -one above the other) or \\
\textbf{Interstitial} (you put the atoms of one in the gaps between the atoms of the other)\\
\\
\underline{About Solubility}:\\
\\
\textbf{No solubility}---- : one phase (ex: Ni, Cu)
\textbf{Has a Solubility limit}---- : 2 phases 
\\

% * <am5616@ic.ac.uk> 2018-01-17T18:02:25.094Z:
% 
% I dont know how to add images on over leaf Vish help
% 
% ^.
 
 \underline{About boundaries and grain shape}:\\
\\
In phases boundaries we have 4 different types:  \\
\textit{coherent, coherency strain, semi-coherent, incoherent};(
coherent : lowest energy, incoherent : highest energy)\\
\\
\vspace{0.75cm}
\\
A grain wants to minimise its \textbf{interfacial energy} :
\begin{itemize}
\item 1 phase : tetrakaidecahedron shape
\item 2 phases with $\alpha$ inside $\beta$
\subitem with $\gamma$ isotropic: sphere shape
\subitem with $\gamma$ non isotropic: depends on $\alpha$ and $\beta$ orientation
\item 2 phases with $\beta$ in the grain boundaries of $\alpha$
\end{itemize}

     \centering   $2\gamma_{\alpha \beta}cos(\theta) = \gamma_{ab}$ (balance of surface tensions) \\
     
(if $\gamma_{\alpha \beta} \leq \frac{1}{2}\gamma_{\alpha \beta}$ then $\beta$ will spread in $\alpha$ grain boundaries)





\end{document}
